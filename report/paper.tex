\documentclass[conference,10pt,letter]{IEEEtran}

\usepackage{url}
\usepackage{amssymb,amsthm}
\usepackage{graphicx,color}

\usepackage{float}

\usepackage{cite}
\usepackage{amsmath}
\usepackage{amssymb}

\usepackage{color, colortbl}
\usepackage{times}
\usepackage{caption}
\usepackage{rotating}
\usepackage{subcaption}

\usepackage{balance}

\newtheorem{theorem}{Theorem}
\newtheorem{example}{Example}
\newtheorem{definition}{Definition}
\newtheorem{lemma}{Lemma}

\newcommand{\XXXnote}[1]{{\bf\color{red} XXX: #1}}
\newcommand{\YYYnote}[1]{{\bf\color{red} YYY: #1}}
\newcommand*{\etal}{{\it et al.}}

\newcommand{\eat}[1]{}
\newcommand{\bi}{\begin{itemize}}
\newcommand{\ei}{\end{itemize}}
\newcommand{\im}{\item}
\newcommand{\eg}{{\it e.g.}\xspace}
\newcommand{\ie}{{\it i.e.}\xspace}
\newcommand{\etc}{{\it etc.}\xspace}
%\newcommand{\em}[1]{\it}

\def\P{\mathop{\mathsf{P}}}
\def\E{\mathop{\mathsf{E}}}

\begin{document}
\sloppy
\title{Process Historian with Cassandra: 24 hour challenge}
\maketitle
\begin{abstract}
Process historian is a time-series database that stores readings 
from, for example, SCADA devices, IoT sensors, and the like. 
Such database should be scalable, durable (should have certain 
resilience to node failures) and support fast write operations.
In this short document we describe our 24 hour challenge in
building such database using Cassandra NoSQL, masterless database.
We use Python Flask as a IoT facing web server. The web server
implements simple REST API for the integration with IoT devices.
\end{abstract}

\input intro.tex 
\input methodology.tex
\input results.tex
\input conclusions.tex

\balance
\bibliographystyle{abbrv}
\bibliography{mybib}

\end{document}
